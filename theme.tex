%%
%% theme.tex for seminaire
%% Made by leblanc <nicuveo@my.epita.fr>
%%


%%%%%%%%%%%%%%%%%%%%%%%%%%%%%%%%%%%%%%%%%%%%%%%%%%%%%%%%%%%%%%%%%%%%%%%%%%%%
%% Original theme

\usepackage{pgf}
\usepackage{listings}
\mode<presentation>{\usetheme{Warsaw}}
\usecolortheme{orchid}



%%%%%%%%%%%%%%%%%%%%%%%%%%%%%%%%%%%%%%%%%%%%%%%%%%%%%%%%%%%%%%%%%%%%%%%%%%%%
%% Colors

% define


%\definecolor{my-color-1}{RGB}{ 60,  64, 132}
%\definecolor{my-color-2}{RGB}{175, 200, 230}
%\definecolor{my-color-3}{RGB}{ 20,  20,  70}
\definecolor{my-color-1}{RGB}{180,  25, 160} % primary
\definecolor{my-color-2}{RGB}{220, 150, 215} % lighter
\definecolor{my-color-3}{RGB}{  0,   0,   0} % unknown
\definecolor{my-color-4}{RGB}{  0,   0,   0} % darker

% \definecolor{a-block-color}{RGB}{225, 210, 240}

\colorlet{block-color}{my-color-2!32!white}
\colorlet{outer-cell}{my-color-2!80!black}

\definecolor{tab-1}{rgb}{0.04,0.34,0.58}
\definecolor{tab-2}{rgb}{0.36,0.56,0.72}
\definecolor{tab-3}{rgb}{0.68,0.78,0.86}


% set

\setbeamercolor{structure}{fg=my-color-1}
\setbeamercolor{alerted text}{fg=my-color-1!56!red}

\setbeamercolor{palette primary}   {fg=white, bg=my-color-1}
\setbeamercolor{palette secondary} {fg=black, bg=my-color-2}
\setbeamercolor{palette tertiary}  {fg=white, bg=my-color-3}
\setbeamercolor{palette quaternary}{fg=white, bg=my-color-4}

% \setbeamercolor{block title}{fg=black,bg=block-color}
% \setbeamercolor{block title example}{use=example text,fg=blue,bg=example text.fg!75!bg}
% \setbeamercolor{block body example}{parent=normal text,use=block title example,bg=block title example.bg!25!bg}
% \setbeamercolor{block title alerted}{use=alerted text,fg=black,bg=a-block-color}

\setbeamercolor{block body}{parent=normal text, use=block title, bg=block-color}
\setbeamercolor{block body alerted}{parent=normal text, use=block title alerted, bg=a-block-color}

% \setbeamercolor{sidebar}{bg=yellow!70}
% \setbeamercolor{palette sidebar primary}   {fg=blue}
% \setbeamercolor{palette sidebar secondary} {fg=blue!75}
% \setbeamercolor{palette sidebar tertiary}  {fg=blue!75}
% \setbeamercolor{palette sidebar quaternary}{fg=red}

\setbeamercolor*{separation line}{}
\setbeamercolor*{fine separation line}{}



%%%%%%%%%%%%%%%%%%%%%%%%%%%%%%%%%%%%%%%%%%%%%%%%%%%%%%%%%%%%%%%%%%%%%%%%%%%%
%% Templates

\defbeamertemplate*{headline}{}
{
  \leavevmode%
  \hbox
  {%
    \begin{beamercolorbox}[wd=0.5\paperwidth,ht=2ex,dp=1.5ex,center]
      {section in head/foot}%
      \usebeamerfont{section in head/foot}
      \insertsectionhead
    \end{beamercolorbox}%

    \begin{beamercolorbox}[wd=0.5\paperwidth,ht=2ex,dp=1.5ex,center]
      {date in head/foot}%
      \usebeamerfont{subsection in head/foot}
      \insertsubsection
    \end{beamercolorbox}%
  }%
  \vskip0pt%
}

\defbeamertemplate{footline}{mine}[2]
{
  \leavevmode%
  \hbox
  {%
    \begin{beamercolorbox}[wd=.160\paperwidth,ht=2.25ex,dp=1.125ex,center]
      {author in head/foot}%
      \usebeamerfont{author in head/foot}
      \pgfimage[mask=#2,interpolate=true,height=1cm]{#1}
    \end{beamercolorbox}%

    \begin{beamercolorbox}[wd=.680\paperwidth,ht=2.25ex,dp=1.125ex,center]
      {title in head/foot}%
      \usebeamerfont{title in head/foot}
      \insertshorttitle
    \end{beamercolorbox}%

    \begin{beamercolorbox}[wd=.160\paperwidth,ht=2.25ex,dp=1.125ex,center]
      {date in head/foot}
      \usebeamerfont{date in head/foot}
      \insertframenumber{} / \inserttotalframenumber{}
    \end{beamercolorbox}%
  }%
  \vskip0pt%
}

\newcommand{\setfooterlogo}[2]
{
  \setbeamertemplate{footline}[mine]{#1}{#2}
}



%%%%%%%%%%%%%%%%%%%%%%%%%%%%%%%%%%%%%%%%%%%%%%%%%%%%%%%%%%%%%%%%%%%%%%%%%%%%
%% Code environnement

\definecolor{hsk-comment}  {gray}{0.30}
\colorlet{hsk-built-ins}{my-color-3!60!blue}
\colorlet{hsk-types}    {my-color-1}
\colorlet{hsk-operators}{my-color-1!40!black}
\colorlet{hsk-keywords} {my-color-1!60!black}
\colorlet{hsk-consts}   {my-color-1!60!blue}
\colorlet{hsk-string}   {my-color-1!60!red}

\lstdefinelanguage{ColorHaskell} {
        basicstyle=\ttfamily\footnotesize,
        sensitive=true,
        morecomment=[l][\color{hsk-comment}\ttfamily\footnotesize]{--},
        morecomment=[s][\color{hsk-comment}\ttfamily\footnotesize]{\{-}{-\}},
        morestring=[b]",
        stringstyle=\color{hsk-string},
        showstringspaces=false,
        numberstyle=\footnotesize
        showspaces=false,
        breaklines=true,
        showtabs=false,
        emph=
        {[1]
                abs,acos,acosh,all,and,any,appendFile,approxRational,asTypeOf,asin,
                asinh,atan,atan2,atanh,basicIORun,break,catch,ceiling,chr,compare,concat,concatMap,
                const,cos,cosh,curry,cycle,decodeFloat,denominator,digitToInt,divMod,drop,
                dropWhile,either,encodeFloat,enumFrom,enumFromThen,enumFromThenTo,enumFromTo,
                error,even,exp,exponent,fail,filter,flip,floatDigits,floatRadix,floatRange,floor,
                fmap,foldl,foldl1,foldr,foldr1,fromDouble,fromEnum,fromInt,fromInteger,fromIntegral,
                fromRational,fst,gcd,getChar,getContents,getLine,head,id,inRange,index,init,intToDigit,
                interact,ioError,isAlpha,isAlphaNum,isAscii,isControl,isDenormalized,isDigit,isHexDigit,
                isIEEE,isInfinite,isLower,isNaN,isNegativeZero,isOctDigit,isPrint,isSpace,isUpper,iterate,
                last,lcm,length,lex,lexDigits,lexLitChar,lines,log,logBase,lookup,map,mapM,mapM_,max,
                maxBound,maximum,maybe,min,minBound,minimum,negate,not,null,numerator,odd,
                or,ord,otherwise,pi,pred,primExitWith,print,product,properFraction,putChar,putStr,putStrLn,
                quotRem,range,rangeSize,read,readDec,readFile,readFloat,readHex,readIO,readInt,readList,readLitChar,
                readLn,readOct,readParen,readSigned,reads,readsPrec,realToFrac,recip,repeat,replicate,return,
                reverse,round,scaleFloat,scanl,scanl1,scanr,scanr1,sequence,sequence_,show,showChar,showInt,
                showList,showLitChar,showParen,showSigned,showString,shows,showsPrec,significand,signum,sin,
                sinh,snd,span,splitAt,sqrt,subtract,succ,sum,tail,take,takeWhile,tan,tanh,threadToIOResult,toEnum,
                toInt,toInteger,toLower,toRational,toUpper,truncate,uncurry,undefined,unlines,until,unwords,unzip,
                unzip3,userError,words,writeFile,zip,zip3,zipWith,zipWith3,listArray,doParse
        },
        emphstyle={[1]\color{hsk-built-ins}},
        emph=
        {[2]
                FilePath,IOError,Bool,Char,Double,Either,Float,IO,Integer,Int,Maybe,Ordering,Rational,Ratio,ReadS,ShowS,String, Word8,InPacket
        },
        emphstyle={[2]\color{hsk-types}},
        emph=
        {[3]
                case,class,data,deriving,do,else,if,import,in,infixl,infixr,instance,let,
                module,of,primitive,then,type,where
        },
        emphstyle={[3]\color{hsk-keywords}\textbf},
        emph=
        {[4]
                quot,rem,div,mod,elem,notElem,seq
        },
        emphstyle={[4]\color{hsk-operators}\textbf},
        emph=
        {[5]
                EQ,False,GT,Just,LT,Left,Nothing,Right,True,Show,Eq,Ord,Num,Enum,Bounded
        },
        emphstyle={[5]\color{hsk-consts}\textbf}
}

\lstnewenvironment{code}
    {\lstset{language=ColorHaskell,basicstyle=\footnotesize\ttfamily}%
      \csname lst@SetFirstLabel\endcsname}
    {\csname lst@SaveFirstLabel\endcsname}
    \lstset{
      basicstyle=\footnotesize\ttfamily,
      flexiblecolumns=false,
      basewidth={0.5em,0.45em},
      literate={+}{{$+$}}1 {/}{{$/$}}1 {*}{{$*$}}1 {=}{{$=$}}1
               {>}{{$>$}}1 {<}{{$<$}}1 {\\}{{$\lambda$}}1
               {\\\\}{{\char`\\\char`\\}}1
               {->}{{$\rightarrow$}}2 {>=}{{$\geq$}}2 {<-}{{$\leftarrow$}}2
               {<=}{{$\leq$}}2 {=>}{{$\Rightarrow$}}2
               {>>}{{>>}}2 {>>=}{{>>=}}2
               {|}{{$\mid$}}1
    }



%%%%%%%%%%%%%%%%%%%%%%%%%%%%%%%%%%%%%%%%%%%%%%%%%%%%%%%%%%%%%%%%%%%%%%%%%%%%
%% Settings

\AtBeginSection[]
{
   \begin{frame}
       \frametitle{Outline}
       \tableofcontents[currentsection,hideothersubsections]
   \end{frame}
}

\setbeamertemplate{navigation symbols}{}
\setbeamercovered{transparent}
%% \setbeamertemplate{background}{%
%%   \parbox[c][\paperheight]{\paperwidth}{%
%%     \vfill \hfill
%%     \includegraphics[width=\paperwidth,height=\paperheight]{bg}
%%     \hfill \vfill
%%   }}

\mode
<all>



%%%%%%%%%%%%%%%%%%%%%%%%%%%%%%%%%%%%%%%%%%%%%%%%%%%%%%%%%%%%%%%%%%%%%%%%%%%%
%% Misc

%\pgfdeclaremask{masklogo}{img/logo}
\newcommand\defaultlogo{\setfooterlogo{img/logo}{}}
\defaultlogo

\bibliographystyle{apalike}

\newcolumntype{I}{!{\vrule width 1.2pt}}
\newlength\savedwidth
\newcommand\whline{\noalign{\global\savedwidth
  \arrayrulewidth\global\arrayrulewidth 1.2pt}
  \hline \noalign{\global\arrayrulewidth
  \savedwidth}
}

\def\extrarowsep{3pt}
\setlength\minrowclearance{3pt}
